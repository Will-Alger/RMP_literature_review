\documentclass[man, 12pt]{apa7}

\usepackage[utf8]{inputenc}
\usepackage[english]{babel}
\usepackage[style=authoryear,backend=bibtex]{biblatex}  
\usepackage{csquotes}
\usepackage{ragged2e}
\usepackage{microtype}

\addbibresource{references.bib}

\title{Exploring the Impacts of Ratemyprofessor.com on Higher Education: A Literature Review}
\shorttitle{RateMyProfesor Impacts: A Literature Review}
\author{William Alger}
\affiliation{Northern Kentucky University}
\note{Special Note (Optional)}

\begin{document}
\maketitle

\section{Abstract}
\begin{abstract}

This literature review aggregates existing research to explore how RateMyProfessor.com influences both the perceptions and actions of students and educators within higher education. Specifically, it explores how RateMyProfessor.com influences students' course choices and educators' pedagogical strategies, as well as their respective self-views and attitudes towards teaching and learning. While the validity and reliability of the reviews on the platform are discussed in broader literature, this review primarily aims to understand the tangible consequences of RateMyProfessor.com on the actions of students and teachers, identifying key areas for future exploration.


\end{abstract}

\section{Introduction}
The rise of digital platforms has profoundly transformed various facets of modern life, including the ways students navigate their academic journey and educators adapt their teaching practices. Amongst these platforms, RateMyProfessor.com stands out as a pivotal tool, enabling students to share their experiences and perspectives on their instructors.

This literature review focuses on the consequential impact of RateMyProfessor.com on the behaviors and actions of students and educators within higher education. We aim to elucidate how the platform directs students' course choices and affects educators' pedagogical approaches and self-awareness. Instead of diving into the platform's validity or reliability, our primary goal is to comprehend the tangible repercussions of RateMyProfessor.com on students' and educators' actions.

Recognizing the significant role that RateMyProfessor.com holds in shaping higher education, especially its sway over the choices and strategies of students and teachers, is vital for informing educational policies, refining pedagogical practices, and enriching the broader academic environment. By offering a holistic overview of the current research, this review endeavors to illustrate the platform's multifaceted influence in the academic landscape and suggest directions for future scholarly exploration.

\subsection{Overview of RateMyProfessor.com}
RateMyProfessor is a website that has been around since 1999. It has long been the leading public platform for students to anonymously leave reviews about their college professors. Many competitors have come and gone, yet RateMyProfessor has stood the test of time. Although RateMyProfessor has remained the most popular platform, many professors and researchers alike have often debated on its true ability to measure teaching effectiveness of faculty. This is partly attributed to the site's lack of verfication of student class enrollment, time since the class was taken, or even University enrollment. This stands in contrast to formal university evaluations, which, although inaccessible for research purposes due to privacy and institutional guidelines, tend to have more stringent requirements for credibility given that only enrolled students who have completed the course can provide feedback.

Despite the limitations of RateMyProfessor, its vast amount of publicly available data - with a little over 2 million professor profiles, nearly 8000 schools, and approximately 22 million reviews - has drawn interest amongst researchers seeking to understand student perceptions in academia. This extensive set of data offers a unique window into students' perceptions regarding teaching efficacy, the challenge level of courses, and overall academic satisfaction, presenting an easily accessible view of higher education from a student's perspective.

\section{Review of Literature}
In the following section, our review of literature will explore the array of research that has been conducted around Ratemyprofessor. This synthesis aims to expose common themes and findings from numerous studies on the platform over the years. Several dimensions of this topic are worth exploring in this review: the impact this site has had on educators and their teaching practices, the influence it exerts on students' course and instructor choices, and the nuances of how it affects both student and educator self-efficacy.

It is noteworthy to mention that this review intentionally omits an exploration of the validity of reviews, their ability to gauge teaching effectiveness, or the overal quality of reviews on RateMyProfessor.com. While such topics have been extensively studied and are a matter of intense debate, they don't necessarily illuminate the tangible repercussions of the platform on the academic environment. Several studies underscore that, regardless of individual perceptions of review validity, the platform's effects are measureable, quantifiable, and have catalyzed shifts in decision-making patterns of both students and educators \textcite{boswell_ratemyprofessors_2016, johnson_i_2014, boswell_effects_2020}. Our focus, therefore, narrows to dissecting the existing literature on the tangible changes RateMyProfessor has induced in the academic landscape.

In acknowledging the limitations and challenges observed in the reviewed studies, we hope to highlight avenues for more robust and nuanced investigations in the future. It is our expectation that by presenting a comprehensive overview of these aspects and revealing uncharted areas of research, we can bring greater clarity to the complex role Ratemyprofessor plays within academia and effectively guide future explorations.

\subsection{Student Engagement with Ratemyprofessor}
Numerous studies have consistently highlighted the widespread usage of RateMyProfessor amongst students. A 2009 study by \textcite{davison_how_2009} surveyed 216 students, revealing that 92\% were aware of RateMyProfessor, 80\% had visited the site more than once, and a notable 95\% deemed it a credible source of information. Further supporting these findings, \textcite{bleske-rechek_ratemyprofessors_2010} showed that amongst their survey of 208 respondents, 84\% of students had visited the site and 23\% had posted a review. More recently, a study conducted by \textcite{chiang_students_2017} with 166 students from a marketing class demonstrated the continued usage of RateMyProfessor: 84.4\% of these students had visited the site within the past two years, and nearly a quarter (24.6\%) had actively participated by posting ratings. It is clear from these samples that Ratemyprofessor has not only drawn the interest of students, but continued to maintain its popularity throughout the years since its launch.

\subsection{Ratemyprofessor as a Decision-making Tool}
The high prevalence of RateMyProfessor usage has encouraged researchers to investigate how this platform impacts academic decisions, notably course selection. Evidence from a survey involving 73 participants, as indicated by \textcite{boswell_effects_2020}, reveals that students' enrollment decisions can be significantly influenced by exposure to university SETs (Student Evaluations of Teaching) or RateMyProfessor evaluations. This base is further strengthened by research from \textcite{orlova_ratemyprofessorscom_2021}, which found that student preferences for specific classes are subject to change based on either positive or negative reviews. Reinforcing this idea, another study by \textcite{johnson_i_2014} demonstrated a positive correlation between professor ratings and course enrollment, thus suggesting the persuasive role of RateMyProfessor in shaping academic choices.

\subsection{Effect of Ratemyprofessor on Student Self-efficacy and perceptions}
The influence of professor ratings on students deeply shapes their educational experience. Not only do these ratings guide course selection, but they also mold students' expectations about their potential success within a class. For instance, \textcite{boswell_effects_2020} found that evaluations, be they positive or negative, have a profound impact on students' self-efficacy. This in turn affects their confidence in grasping the course material and meeting expected learning outcomes. A similar sentiment is echoed by \textcite{kowai-bell_rate_2011}, who demonstrated that evaluations from Ratemyprofessor notably influence students' perceived control, grade expectations, and overall attitude toward the class.

Further deepening the significance of these ratings, \textcite{scherr_single_2013} emphasized how even a solitary negative comment on Ratemyprofessor can dramatically alter students' perception of the course environment, even if other ratings are predominantly positive. This demonstrates the potent power of online reviews in shaping student perspectives, which can then ripple into their academic performance, course choices, and overall contentment with their education.

Building on this, \textcite{reber_perceptual_2017} discovered that students exposed to positive evaluations not only reported higher engagement in their classes but also better performance as compared to their counterparts who encountered negative reviews. This suggests a self-fulfilling prophecy at play: students who encounter positive evaluations develop a heightened sense of confidence and engagement, leading to superior academic outcomes.

\subsection{Effect of RateMyProfessor on Teacher Self-efficacy}
The impact of RateMyProfessor is not confined to students alone; it also presents implications for teacher as well. Even though professors often view university SETs as a more accurate form of evaluation compared to RateMyProfessor evaluations, they are still significantly influenced by the latter, despite perceiving it as less accurate \textcite{boswell_ratemyprofessors_2016}. \textcite{boswell_ratemyprofessors_2016} further found that teachers' awareness and perception of their online ratings influenced their sense of professional competence and effectiveness in the classroom. Negative reviews, in particular, were associated with decreased teacher self-efficacy, which could alter teaching practices and pedagogical choices over time. Further discussion is warrented whether this could lead educators to adopt a 'crowd-pleasing' approach rather than focusing on educational quality.

% Possible study of "Do teachers improve over time?" transition this into further research

\section{Discussion}

The literature reviewed herein provides compelling evidence of the significant influence RateMyProfessor.com exerts on higher education, particularly concerning students' course choices and perceptions of course difficulty and faculty effectiveness. Simultaneously, it showcases how educators' pedagogical strategies and self-perceptions are affected by their online ratings. 

RateMyProfessor.com's influence is not just limited to immediate behaviors but may also have long-term effects that shape the educational journey of students and the professional development of educators. For instance, the reported impact on student self-efficacy could potentially affect students' academic performance, future course selections, and overall satisfaction with their education. A similar long-lasting effect might be observed amongst educators whose teaching styles may evolve based on feedback received on this platform.

However, these assertions remain speculative until further empirical research is conducted. While existing studies offer insights into the immediate consequences of RateMyProfessors.com on the behaviors and attitudes of students and educators, they tend to overlook its longitudinal effects. Evaluating changes in students' academic trajectories or shifts in teachers' pedagogical methods over time could provide a more comprehensive understanding of the platform's influence. 

Future research should also investigate whether changes in teachers' ratings over time correspond to modifications in their teaching practices. Such investigation can uncover whether and how educators use the feedback from RateMyProfessor.com for their professional improvement. This would help evaluate the potential of RateMyProfessor.com as a tool for bettering teaching quality while also addressing concerns about its possible misuse to merely placate students rather than genuinely enhancing teaching efficacy.

In conclusion, we suggest that subsequent research delve into the long-term impacts of RateMyProfessor.com on higher education, thereby bridging the current gap in understanding. These investigations will be essential in fully capturing the multifaceted influence of this platform on shaping the dynamics of teaching and learning within academia.

\section{Summary}
This literature review consolidated existing research to analyze the influence of RateMyProfessor.com on both students' and educators' perspectives and behaviors within higher education. The findings underscore RateMyProfessor.com's significant role in shaping students' course selections, self-efficacy, and perceptions of their education, as well as influencing educators' pedagogical strategies and professional self-perceptions.

The widespread use of RateMyProfessor.com among students is evident, with it serving as an important platform for decision-making regarding course selection. The site's evaluations significantly impact students' confidence and anticipated success in the classroom. Even a single negative comment can notably affect students' perception of the course climate.

For educators, awareness of their online ratings on RateMyProfessor.com has been found to contribute to their sense of professional competence. Negative reviews, in particular, can lead to decreased teacher self-efficacy, potentially impacting teaching practices and pedagogical choices.

While this review did not delve into the validity or reliability of the ratings on RateMyProfessor.com, the measurable effects of the platform on students' and teachers' actions are evident. By elucidating these effects, this review contributes to our understanding of how digital platforms can shape higher education and points to areas for future research.







\printbibliography
\end{document}