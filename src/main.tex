\documentclass[man, 12pt]{apa7}

\usepackage[utf8]{inputenc}
\usepackage[english]{babel}
\usepackage[style=authoryear,backend=bibtex]{biblatex}  
\usepackage{csquotes}
\usepackage{ragged2e}
\usepackage{microtype}

\addbibresource{references.bib}

\title{Exploring the Impacts of Ratemyprofessor.com on Higher Education: A Literature Review}
\shorttitle{RateMyProfesor Impacts: A Literature Review}
\author{William Alger}
\affiliation{Northern Kentucky University}
\note{ My sincere thanks to Dr. Nicholas Caporusso for his invaluable guidance on this project.}


\begin{document}
\maketitle

\section{Abstract}
\begin{abstract}
-This literature review explores the multifaceted role of RateMyProfesor.com in shaping the perceptions and actions of students and teachers in higher education. 
The objective of this review is to gain a deeper understanding of the platform, evaluate its tangible impact on students and teachers, and potentially identify strategies for improving its effectiveness as an educational tool. Findings reveal that RateMyProfessor's widespread usage amongst students plays a moderate role in shaping perceptions, academic decision-making, and course selection. One study highlights how online reviews present on RateMyProfessor are capable of shaping students' self-efficacy, perceived control, anticipated grades, and expectations of the course environment, often setting a precedent before even stepping foot in the classroom. Furthermore, teachers with positive reviews have the potential to heighten students' expectations of success, resulting in increased engagement and classroom performance. On the contrary, exposure to negative reviews may dampen these expectations and lead to poorer performance and engagement. To one extreme, an article revealed that even a single negative comment has the potential to shift students' perception despite an overall positive rating. This impact extends to professors as well; despite a significant preference for formal university evaluations, professors are equally influenced by their RateMyProfessor ratings. Reviews on RateMyProfessor have the ability to alter a teachers sense of professional competence and teaching methodologies. Long term effects could be detrimental to pedagogical choices. The dual effect on teachers and students underscores the need for a balanced approach to student feedback systems. In light of the sources reviewed, this paper suggests enhancements to RateMyProfessor, primarily focusing on refining its feedback mechanism. As we conclude, we emphasize the continued need for additional research into the role these platforms play in educational quality.
\end{abstract}


% Introduction to the Digital Era: Discuss how the digital era has transformed various aspects of life including social interactions, online businesses, telemedicine, entertainment, and notably, education.

% Rise of Online Platforms and Influence in Education: Introduce the rise of online platforms like RateMyProfessor.com and its transformational impact on the educational sector.

% Purpose of the Literature Review: Outline the purpose of the literature review - which is to explore the influence of RateMyProfessor.com on both students and teachers in higher education.

% The Consequences and Impacts: Highlight the specific impacts the platform has on academic decision-making, course selection, self-efficacy, anticipated grades, and overall perspective on the course environment.

% Importance of Recognizing the Role of RateMyProfessor.com: Emphasize the significance of acknowledging the role and sway of RateMyProfessor.com over choices and strategies of students and teachers.

% Aim of the Study: State the aim of the study, focusing on informing educational policies, refining pedagogical practices, and enriching the broader academic environment.

% Overview of the Review: Briefly describe what the review will cover and suggest directions for future scholarly exploration.

\section{Introduction}
The rise of the digital era has transformed many facets of everyday life, such as social interactions, online businesses, telemedicine, entertainment, and most notably, education. Higher education has seen a noticeable shift with the rise of online platforms that aim to augment the traditional learning mechanisms. In particular, RateMyProfessor.com has emerged as a pivotal tool in shaping this landscape, influencing not only how students navigate their academic journey but also how educators adapt their teaching practices.

RateMyProfessor.com provides a platform where students share their experiences and perspectives on their instructors. This interaction guides students' course selections and shapes their expectations about the learning environment. Simultaneously, it gives educators feedback that can prompt reflection and refinement in their pedagogical approaches.

Recognizing the significant role that RateMyProfessor.com plays in higher education is integral for informing educational policies, refining pedagogical practices, and enriching the broader academic environment. This literature review seeks to explore the consequential impacts of RateMyProfessor.com within higher education, focusing particularly on the behaviors, actions, and perceptions of students and educators.

To achieve this aim, we offer a comprehensive overview of the current research on the effects of RateMyProfessor.com. Presenting a balanced view of its potential benefits and downsides, we describe how the platform shapes the academic landscape in multiple ways. Our goal is to highlight where further scholarly exploration can contribute to a deeper understanding of these dynamics, thereby suggesting productive directions for future research.

\subsection{Overview of RateMyProfessor.com}
RateMyProfessor is a website that has been around since 1999. It has long been the leading public platform for students to anonymously leave reviews about their college professors. Many competitors have come and gone, yet RateMyProfessor has stood the test of time. Although RateMyProfessor has remained the most popular platform, many professors and researchers alike have often debated on its true ability to measure teaching effectiveness of faculty. This can be partly attributed to the platform's absence of mechanisms to verify student class enrollment, recency of the class taken, or even university affiliation. This stands in contrast to formal university evaluations. Despite being inaccessible for research due to privacy and institutional guidelines, university SETs often have stricter credibility requirements: only enrolled students who completed the course are able to provide feedback.

Despite the limitations of RateMyProfessor, its vast amount of publicly available data, which includes a little over 2 million professor profiles, nearly 8000 schools, and approximately 22 million reviews, has attracted the attention of researchers aiming to understand student perceptions in academia. This extensive set of data offers a unique window into students' perceptions regarding teaching efficacy, the challenge level of courses, and overall academic satisfaction, presenting an easily accessible view of higher education from a student's perspective.

\section{Review of Literature}
In the following section, our review of literature will explore the array of research that has been conducted around RateMyProfessor. Our objective is to uncover recurring themes and significant findings from these studies over the years. We aim to explore several facets of this topic in our review: the platform’s influence on educators and their pedagogical methods, its sway over students' course and instructor selections, and the subtle ways it impacts both student and educator self-efficacy.

Setting the scope of our analysis, we narrow our focus to the influence that RateMyProfessor exerts on academia rather than its validity as an evaluation tool. As highlighted by \textcite{rosen_correlations_2018}, experts opinions about RateMyProfessor’s accuracy in assessing teaching effectiveness vary widely. Despite the varied opinions on review credibility, research overwhelmingly demonstrates that these evaluations significantly influence our academic environment.

Given this context, it is important to clarify our investigation's focus within the broader discourse around RateMyProfessor. This review intentionally omits an exploration of literature around the platform's review validity, ability to measure teaching effectiveness, or potential biases inherent in its reviews (see for example, \textcite{reid_role_2010,hartman_what_2013, azab_analysing_2016, boring_gender_2017, rosen_correlations_2018, baker_quantcrit_2019, gordon_role_2021}). While this perspective is crucial to comprehend what students value in an educator and how they evaluate them, it does not directly highlight the shift in choices and actions in academia as a result of those reviews. Numerous studies emphasize that despite differing views on review credibility or potential bias, the platform's effects are significant, measurable, and have instigated shifts in decision-making patterns amongst students and educators alike \textcite{johnson_i_2014, boswell_ratemyprofessors_2016, boswell_effects_2020}.

Consequently, this review's primary goal is to understand the concrete impacts of RateMyProfessor.com on students' and educators' behavior, rather than dissecting the platform's validity, reliability, or review biases. Thus, we narrow our scope to scrutinize existing literature on tangible shifts in the academic landscape that can be attributed to RateMyProfessor.

In acknowledging the limitations and challenges observed in the reviewed studies, we hope to highlight avenues for more robust and nuanced investigations in the future. It is our expectation that by presenting a comprehensive overview of these aspects and revealing uncharted areas of research, we can bring greater clarity to the complex role RateMyProfessor plays within academia and effectively guide future explorations.

\subsection{Student Engagement with RateMyProfessor}
Numerous studies have consistently highlighted the widespread usage of RateMyProfessor amongst students. A 2009 study by \textcite{davison_how_2009} surveyed 216 students, revealing that 92\% were aware of RateMyProfessor, 80\% had visited the site more than once, and a notable 95\% deemed it a credible source of information. Further supporting these findings, \textcite{bleske-rechek_ratemyprofessors_2010} showed that amongst their survey of 208 respondents, 84\% of students had visited the site and 23\% had posted a review. More recently, a study conducted by \textcite{chiang_students_2017} with 166 students from a marketing class demonstrated the continued usage of RateMyProfessor: 84.4\% of these students had visited the site within the past two years, and nearly a quarter (24.6\%) had actively participated by posting ratings. It is clear from these samples that RateMyProfessor has not only drawn the interest of students, but continued to maintain its popularity throughout the years since its launch.

\subsection{RateMyProfessor as a Decision-making Tool}
The high prevalence of RateMyProfessor usage has encouraged researchers to investigate how this platform impacts academic decisions, notably course selection. Evidence from a survey involving 73 participants, as indicated by \textcite{boswell_effects_2020}, reveals that students' enrollment decisions can be significantly influenced by exposure to university SETs (Student Evaluations of Teaching) or RateMyProfessor evaluations equally. This base is further strengthened by research from \textcite{orlova_ratemyprofessorscom_2021}, which found that student preferences for specific classes are subject to change based on either positive or negative RateMyProfessor evaluations. Reinforcing this idea, another study by \textcite{johnson_i_2014} demonstrated a positive correlation between professor ratings and course enrollment. Collectively these studies suggest that RateMyProfessor plays a moderate role in shaping academic decisions.

\subsection{Effect of Ratemyprofessor on Student Self-efficacy and Perceptions}
The influence of professor ratings on students has the potential to shape their classroom experience before it has even started. Not only do these ratings guide course selection, but they also mold students' expectations about their potential success within a class. For instance, \textcite{boswell_effects_2020} found that evaluations, be they positive or negative, have a profound impact on students' self-efficacy. This in turn affects their confidence in grasping the course material and meeting expected learning outcomes. A similar sentiment is echoed by \textcite{kowai-bell_rate_2011}, who demonstrated that evaluations from RateMyProfessor notably influence students' perceived control, grade expectations, and overall attitude toward the class.

Revealing the importance of these ratings further, \textcite{scherr_single_2013} emphasized how even a solitary negative comment on RateMyProfessor can dramatically alter students' perception of the course environment, even if other ratings are predominantly positive. This demonstrates the potent power of online reviews in shaping student perspectives, which can then ripple into their academic performance, course choices, and overall contentment with their education.

Building on this, \textcite{reber_perceptual_2017} discovered that students exposed to positive evaluations not only reported higher engagement in their classes but also better performance as compared to their counterparts who encountered negative reviews. This suggests a self-fulfilling prophecy at play: students who encounter positive evaluations develop a heightened sense of confidence and engagement, leading to superior academic outcomes.

\subsection{Effect of RateMyProfessor on Teacher Self-efficacy}
The impact of RateMyProfessor is not confined to students alone; it also presents implications for teacher as well. Even though professors often view university SETs as a more accurate form of evaluation compared to RateMyProfessor evaluations, they are still significantly influenced by the latter, despite perceiving it as less accurate \textcite{boswell_ratemyprofessors_2016}. \textcite{boswell_ratemyprofessors_2016} further found that teachers' awareness and perception of their online ratings influenced their sense of professional competence and effectiveness in the classroom. Negative reviews, in particular, were associated with decreased teacher self-efficacy, which could alter teaching practices and pedagogical choices over time. Further discussion is warrented whether this could lead educators to adopt a 'crowd-pleasing' approach rather than focusing on educational quality.

% Possible study of "Do teachers improve over time?" transition this into further research

\section{Discussion}

The literature reviewed herein provides compelling evidence of the significant influence RateMyProfessor.com exerts on higher education, particularly concerning students' course choices and perceptions of course difficulty and faculty effectiveness. Simultaneously, it showcases how educators' pedagogical strategies and self-perceptions are affected by their online evaluations. 

RateMyProfessor.com's influence is not just limited to immediate behaviors but may also have long-term effects that shape the educational journey of students and the professional development of educators. For instance, the reported impact on student self-efficacy could potentially affect students' academic performance, future course selections, and overall satisfaction with their education. A similar long-lasting effect might be observed amongst educators whose teaching styles may evolve based on feedback received on this platform.

However, these assertions remain speculative until further empirical research is conducted. While existing studies offer insights into the immediate consequences of RateMyProfessors.com on the behaviors and attitudes of students and educators, they tend to overlook its longitudinal effects. Evaluating changes in students' academic trajectories or shifts in teachers' pedagogical methods over time could provide a more comprehensive understanding of the platform's influence. 

Future research might delve into whether teachers on RateMyProfessor improve with time, or if accumulating negative reviews contribute to lower self-efficacy, leading to poorer subsequent evaluations. This kind of investigation could shed light on how educators utilize feedback from the platform for their professional development and assess whether RateMyProfessor can serve as a tool for enhancing teaching quality beyond simply appeasing students.

In conclusion, we suggest that subsequent research delve into the long-term impacts of RateMyProfessor.com on higher education, thereby bridging the current gap in understanding. These investigations will be essential in fully capturing the multifaceted influence of this platform on shaping the dynamics of teaching and learning within academia.

\section{Summary}
This literature review consolidated existing research to analyze the influence of RateMyProfessor.com on both students' and educators' perspectives and behaviors within higher education. The findings underscore RateMyProfessor.com's significant role in shaping students' course selections, self-efficacy, and perceptions of their education, as well as influencing educators' pedagogical strategies and professional self-perceptions.

The widespread use of RateMyProfessor.com amongst students is evident, with it serving as an important platform for decision-making regarding course selection. The site's evaluations significantly impact students' confidence and anticipated success in the classroom. Even a single negative comment can notably affect students' perception of the course climate.

For educators, awareness of their online ratings on RateMyProfessor.com has been found to contribute to their sense of professional competence. Negative reviews, in particular, can lead to decreased teacher self-efficacy, potentially impacting teaching practices and pedagogical choices.

While this review did not delve into the validity or reliability of the ratings on RateMyProfessor.com, the measurable effects of the platform on students' and teachers' actions are evident. By elucidating these effects, this review contributes to our understanding of how digital platforms can shape higher education and points to areas for future research.







\printbibliography
\end{document}