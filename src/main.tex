\documentclass[doc, 12pt, twocolumn]{apa7}

\usepackage[utf8]{inputenc}
\usepackage[english]{babel}
\usepackage[style=authoryear,backend=bibtex]{biblatex}  
\usepackage{csquotes}
\usepackage{ragged2e}
\usepackage{microtype}

\addbibresource{references.bib}

\title{Exploring the Impacts of Ratemyprofessor.com on Higher Education: A Literature Review}
\shorttitle{Ratemyprofessor: A Literature Review}
\author{William Alger}
\affiliation{Northern Kentucky University}
\note{Special Note (Optional)}

\begin{document}
\maketitle

\section{Abstract}
\begin{abstract}
-This literature review examines the impact of Ratemyprofessor.com, a popular online academic review platform, on higher education. The paper synthesizes existing research to understand the site's influence on students' academic decisions and perceptions of educators, as well as its potential biases and limitations. The study highlights the site's enduring popularity amongst students, with substantial use for decision-making regarding course and instructor choices. However, despite widespread usage, questions regarding the platform's validity and reliability persist due to unverified reviews. This review aims to clarify Ratemyprofessor.com's multifaceted role within academia and identify areas for future investigation.
\end{abstract}

\section{Introduction}
The rise of digital platforms has drastically altered many aspects of our lives, including how students make academic decisions and shape their college experience. One of the most popular and widely utilized websites, Ratemyprofessor.com, has become an influential tool for students to voice their experiences and opinions on their educators.

This literature review aims to explore the themes that have emerged in research conducted on Ratemyprofessor thus far. It seeks to delve into its influence on student academic decisions, perceptions of educators, and the overall college experience. Furthermore, this review aims to spotlight how research has demonstrated the effects Ratemyprofessor has had on educators themselves, specifically regarding self-efficacy in teaching and modifications made to the classroom environment as the result of student evaluations. Understanding the substantial role that Ratemyprofessor.com plays in higher education, particularly its influence on the decision-making processes of both students and educators, is crucial to shape educational policies, enhance teaching methodologies, and improve overall academic experiences. Thus, by providing a comprehensive synthesis of existing research, we aim to clarify how this tool shapes academia and guide future investigations into this critical aspect of contemporary education.


\subsection{Overview of Ratemyprofessor.com}
Ratemyprofessor is a website that has been around since 1999. It has long been the leading public platform for students to anonymously leave reviews about their college professors. Many competitors have come and gone, yet Ratemyprofessor has stood the test of time. Although Ratemyprofessor has remained the most popular platform, many professors, researchers, and students alike have often debated on its true value. This could be partly attributed to the site's lack of verification process, as it doesn't mandate reviewers to confirm their enrollment in the class or even their status as a college student. This stands in contrast to formal university evaluations, which, although inaccessible for research purposes due to privacy and institutional guidelines, tend to have more stringent requirements for credibility given that only enrolled students who have completed the course can provide feedback.

Despite the limitations of Ratemyprofessor, its vast amount of publicly available data - with a little over 2 million professor profiles and approximately 22 million reviews - has drawn interest amongst researchers seeking to understand student perceptions in academia. This extensive set of data offers a unique window into students' perceptions regarding teaching efficacy, the challenge level of courses, and overall academic satisfaction, presenting an easily accessible view of higher education from a student's perspective.

\section{Review of Literature}
In the following section, our review of literature will explore the array of research that has been conducted around Ratemyprofessor. This synthesis aims to expose common themes and findings from numerous studies on the platform over the years. Several dimensions of this topic are worth exploring: the impact this site has had on educators and their teaching practices, the influence it exerts on students' course and instructor choices, potential biases inherent in the reviews, and other factors at play. In acknowledging the limitations and challenges observed in the reviewed studies, we hope to highlight avenues for more robust and nuanced investigations in the future. It is our expectation that by presenting a comprehensive overview of these aspects and revealing uncharted areas of research, we can bring greater clarity to the complex role Ratemyprofessor plays within academia and effectively guide future explorations.

\subsection{Student Engagement with Ratemyprofessor}
Numerous studies have consistently highlighted the widespread usage of RateMyProfessor amongst students. A 2009 study by \textcite{davison_how_2009} surveyed 216 students, revealing that 92\% were aware of RateMyProfessor, 80\% had visited the site more than once, and a notable 95\% deemed it a credible source of information. Further supporting these findings, \textcite{bleske-rechek_ratemyprofessors_2010} showed that amongst their survey of 208 respondents, 84\% of students had visited the site and 23\% had posted a review. More recently, a study conducted by \textcite{chiang_students_2017} with 166 students from a marketing class demonstrated the continued usage of RateMyProfessor: 84.4\% of these students had visited the site within the past two years, and nearly a quarter (24.6\%) had actively participated by posting ratings. It is clear from these samples that Ratemyprofessor has not only drawn the interest of students, but continued to maintain its popularity throughout the years since its launch.

\subsection{Ratemyprofessor as a Decision-making Tool}
With such widespread usage of RateMyProfessor demonstrated, researchers have been prompted to explore the extent to which this platform guides their academic decisions such as course selection. In a survey of 73 participants, \textcite{boswell_effects_2020} concluded that exposure to either university SETs (Student Evaluations of Teaching) or Ratemyprofessor evaluations could influence students choice to enroll in a course at all. This is supported by \textcite{johnson_i_2014} who found that course enrollment was positively correlated with professor ratings.


\subsection{Effect of Ratemyprofessor on Student Self-efficacy}
The influence of professor ratings on students extends beyond course selection to their anticipated success in class. \textcite{boswell_effects_2020} discovered that evaluations, either positive or negative, significantly influenced students' self-efficacy, thereby affecting their confidence in the course content and expected learning outcomes. This finding is echoed by \textcite{kowai-bell_rate_2011}, which demonstrated a positive influence of Ratemyprofessor evaluations on students' perceived control, grade expectancy, and attitude toward the class.

Moreover, as \textcite{scherr_single_2013} highlighted, even a single negative comment on Ratemyprofessor can significantly impact students' perception of the course climate, despite accompanying positive ratings. These findings collectively underscore how Ratemyprofessor can shape the education experience at a deep level. They reflect the power of online reviews to significantly sway student perceptions and expectations, potentially infleuncing their academic performance, course decisions, and satisfaction with their education.

% add section here about li2013

\subsection{Effect of RateMyProfessor on Teacher Self-efficacy}
The impact of RateMyProfessor is not confined to students alone; it also presents implications for teacher self-efficacy. 



\textcite{smith_teacher_2018} found that teachers' awareness and perception of their online ratings significantly influenced their sense of professional competence and effectiveness in the classroom. Negative reviews, in particular, were associated with decreased teacher self-efficacy, which could alter teaching practices and pedagogical choices over time.

Concerningly, this influence was noted by \textcite{jones_pedagogical_2019} who discovered that the fear of negative evaluations could lead teachers to adopt a 'crowd-pleasing' approach rather than focusing on educational quality. This highlights an unintended consequence of RateMyProfessor, where the platform's power to shape teacher self-efficacy may inadvertently affect the quality of education. Therefore, while tools like RateMyProfessor can provide valuable feedback, they also underscore the importance of incorporating more comprehensive and balanced evaluation systems that accurately capture teaching effectiveness.









\printbibliography
\end{document}