\documentclass[twocolumn, doc,12pt]{apa7}
% \documentclass[man,12pt]{apa7}

\usepackage[utf8]{inputenc}
\usepackage[english]{babel}
\usepackage[style=authoryear,backend=bibtex]{biblatex}  
\usepackage{csquotes}
\usepackage{ragged2e}
\usepackage{microtype}

\addbibresource{references.bib}

\title{Exploring the Impacts of Ratemyprofessor.com on Higher Education: A Literature Review}
\shorttitle{RateMyProfessor Impacts: A Literature Review}
\author{William Alger}
\affiliation{Northern Kentucky University}
\note{My sincere thanks to Dr. Nicholas Caporusso for his invaluable guidance on this project, and to the Honors College for providing me with this opportunity}

\begin{document}
\maketitle

\section{Abstract}
\begin{abstract}
-This literature review explores the multifaceted role of RateMyProfessor.com in shaping the perceptions and actions of students and teachers in higher education. 
The objective of this review is to gain a deeper understanding of the platform, evaluate its tangible impacts on students and teachers, and potentially identify strategies for improving its effectiveness as an educational tool. Findings reveal that RateMyProfessor's widespread usage amongst students plays a moderate role in shaping perceptions, academic decision-making, and course selection. One study highlights how online reviews present on RateMyProfessor are capable of shaping students' self-efficacy, perceived control, anticipated grades, and expectations of the course environment, often setting a precedent before even stepping foot in the classroom. Furthermore, teachers with positive reviews have the potential to heighten students' expectations of success, resulting in increased engagement and classroom performance. On the contrary, exposure to negative reviews may dampen these expectations and lead to poorer performance and engagement. To one extreme, an article revealed that even a single negative comment has the potential to shift students' perceptions despite an overall positive rating held by a professor. This impact extends to professors as well; despite a significant preference for formal university evaluations, professors are equally influenced by their RateMyProfessor ratings. Reviews on RateMyProfessor have the ability to alter a teachers sense of professional competence and teaching methodologies. Long term effects could be detrimental to pedagogical choices. The dual effect on teachers and students underscores the need for a balanced approach to student feedback systems. In light of the sources reviewed, this paper suggests enhancements to RateMyProfessor, primarily focusing on refining its feedback mechanism. As we conclude, we emphasize the continued need for additional research into the role these platforms play in educational quality.
\end{abstract}

\section{Introduction}
% Background and Context
The rise of the digital era has transformed many facets of everyday life, including social interactions, online businesses, telemedicine, entertainment, and notably, education. Higher education has seen a noticeable shift with the rise of online platforms that aim to augment traditional learning mechanisms and tools. In particular, RateMyProfessor.com has emerged as a pivotal tool in shaping this landscape, influencing not only how students navigate their academic journey but also how educators adapt their teaching practices.

% Significance of the Topic
RateMyProfessor is more than a mere outlet for disgruntled students. The platform serves as one of the most widely used tools for student feedback on professors and schools. It has the potential to influence both the classes students choose to take, and the way teachers receive feedback, potentially altering pedagogical choices over time.  The platform's extensive scale and sustained use underscore the ongoing necessity for more educational tools in academia. Understanding the reach and impact of RateMyProfessor is vital for informing educational policies, guiding the development of newer, improved tools for students and teachers, and enriching the overall academic environment.

Guiding the analysis for this literature review is the assumption that RateMyProfessor, due to its extensive reach and influence within higher education, impacts both students and educators in a multitued of ways. Subsequently, this review will explore the tangible impacts of RateMyProfessor.com. We aim to focus specifically on its influence on the behaviors, actions, and perceptions of students and educators in and out of the classroom. Lastly, we aim to highlight areas demanding further research and identify potential strategies to enhance RateMyProfessor's effectiveness as an educational tool, thereby improving outcomes for both students and educators.


\section{Overview of RateMyProfessor.com}
% Brief History
RateMyProfessor is a website that has been around since 1999. It has long been the leading public platform for students to leave anonymous reviews about their college professors. Many competitors have come and gone, yet RateMyProfessor has stood the test of time. 
% Platform Features
RateMyProfessor enables any registered user to submit reviews for professors or schools. Users can share comments and rate various aspects, including the difficulty of a professor's course and their overall teaching quality. Similarly, users can evaluate specific facets of schools, such as social life and academic opportunities. Each metric rated on a 5-point scale.

%  User Base
While educators might use RateMyProfessor to gain insights into their teaching practice, the platform is predominantly catered towards students. According to the website's guidelines, their mission aims at: "providing a safe forum to share classroom experiences to help fellow students make critical education choices." This observation is further supported by the general layout of the platform, which closely mirrors a product review page more than a pedagogical tool for educators to enhance their teaching methods and student engagement strategies.

%  Criticisms and Controversies
Despite RateMyProfessors' continued popularity, its ability to be an effective measure of teacher effectiveness has long been contested by researchers \textcite{rosen_correlations_2018}. This skepticism could stem from the platform's lack of mechanisms for verifying student class enrollment, time the class was taken, or even university affiliation. Moreover, the anonymous nature of the platform's online reporting system leaves it suspetible to a wide range of biases.

%  Research interest and Data availability
Regardless of the limitations of RateMyProfessor, its vast amount of publicly availabile data -- encompassing over two million professor profiles, nearly 8000 schools, and tens of millions of reviews -- has long-drawn the attention of researchers seeking to understand student perceptions in academia. This stands in contrast to formal university evaluations, which, due to privacy and institutional guidelines, are often challenging to acquire or inaccessible for research purposes. The considerable scale of RateMyProfessor, coupled with its long-standing presence in the academic landscape, makes the platform a particularly worthwhile subject for exploration. Thoroughly understanding the nature of its influence is crucial for gaining insights into how we can develop a more effective tool for both students and teachers in higher education.

\section{Review of Literature}
In the following section, our review of literature will explore the array of research that has been conducted around RateMyProfessor. Our objective is to uncover recurring themes and significant findings from these studies over the years. We aim to explore several facets of this topic in our review: the platform’s influence on educators and their pedagogical methods, its sway over students' course and instructor selections, and the subtle ways it impacts both student and educator self-efficacy.

Setting the scope of our analysis, this review intentionally omits an exploration of literature around the platform's review validity, ability to measure teaching effectiveness, or potential biases inherent in its reviews (see for example, \textcite{reid_role_2010,hartman_what_2013, azab_analysing_2016, boring_gender_2017, rosen_correlations_2018, baker_quantcrit_2019, gordon_role_2021}). Experts opinions about RateMyProfessor’s accuracy in assessing teaching effectiveness vary widely. While considering this perspective is useful for gaining a deeper understanding about student perceptions and biases in higher education, this discussion does not contribute to the goal of this review.  Numerous studies emphasize that despite differing views on review credibility or potential bias, the platform's effects are significant, measurable, and have instigated shifts in decision-making patterns amongst students and educators alike \textcite{johnson_i_2014, boswell_ratemyprofessors_2016, boswell_effects_2020}. By acknowledging the limitations and methodologies of the studies reviewed, we aim to highlight gaps in the existing literature and outline potential avenues for future research.

\section{Student Engagement with RateMyProfessor}
Numerous studies have consistently highlighted the widespread usage of RateMyProfessor amongst students. A 2009 study by \textcite{davison_how_2009} surveyed 216 students, revealing that 92\% were aware of RateMyProfessor, 80\% had visited the site more than once, and a notable 95\% deemed it a credible source of information. Further supporting these findings, \textcite{bleske-rechek_ratemyprofessors_2010} showed that amongst their survey of 208 respondents, 84\% of students had visited the site and 23\% had posted a review. More recently, a study conducted by \textcite{chiang_students_2017} with 166 students from a marketing class demonstrated the continued usage of RateMyProfessor: 84.4\% of these students had visited the site within the past two years, and nearly a quarter (24.6\%) had actively participated by posting ratings. It is clear from these samples that RateMyProfessor has not only drawn the interest of students, but continued to maintain its popularity throughout the years since its launch.

\section{RateMyProfessor as a Decision-making Tool}
The high prevalence of RateMyProfessor usage has encouraged researchers to investigate how this platform impacts academic decisions, notably course selection. Building upon this idea, a study by \textcite{johnson_i_2014} demonstrated a positive correlation between professor ratings and course enrollment. This base is further strengthened by research from \textcite{orlova_ratemyprofessorscom_2021}, which found in a sample of 51 participants that student preferences for specific classes are subject to change based on either positive or negative RateMyProfessor evaluations. Similarly, evidence from a survey involving 73 participants, as indicated by \textcite{boswell_effects_2020}, reveals that students' enrollment decisions can be significantly influenced by exposure to university SETs (Student Evaluations of Teaching) or RateMyProfessor evaluations equally. These studies highlight two significant aspects of RateMyProfessor's impact: firstly, its power to shape student class preferences, and secondly, its influence on course enrollment numbers. Although the sample sizes for these studies are relatively small, they provide moderate evidance that RateMyProfessor plays a role as a course selection tool for students.

\section{Effect of RateMyProfessor on Student Self-efficacy and Perceptions}
The influence of professor ratings on students has the potential to shape their educational experience before even stepping foot in the classroom. Not only do these ratings guide course selection, but they also mold students' expectations about their potential success within a class. For instance, \textcite{boswell_effects_2020} found that evaluations, be they positive or negative, have a profound impact on students' self-efficacy. This in turn affects their confidence in grasping the course material and meeting expected learning outcomes. A similar sentiment is echoed by \textcite{kowai-bell_rate_2011}, who demonstrated that evaluations from RateMyProfessor notably influence students' perceived control, grade expectations, and overall attitude toward the class.

Revealing the importance of these ratings further, \textcite{scherr_single_2013} emphasized how even a solitary negative comment on RateMyProfessor can dramatically alter students' perception of the course environment, even if other ratings are predominantly positive. This demonstrates the potent power of online reviews in shaping student perspectives, which can then ripple into their academic performance, course choices, and overall contentment with their education.

Building on this, \textcite{reber_perceptual_2017} discovered that students exposed to positive evaluations not only reported higher engagement in their classes but also better performance as compared to their counterparts who encountered negative reviews. Findings from this study support the results presented in \textcite{boswell_effects_2020}. These results collectively suggest a self-fulfilling prophecy at play: students who encounter positive evaluations develop a heightened sense of confidence and engagement, leading to superior academic outcomes.

\section{Effect of RateMyProfessor on Professors}
The impact of RateMyProfessor is not confined to students alone; it also presents implications for professors as well. Even though professors often view university SETs as a more accurate form of evaluation compared to RateMyProfessor evaluations, they are still significantly influenced by the latter, despite perceiving it as less accurate \textcite{boswell_ratemyprofessors_2016}. \textcite{boswell_ratemyprofessors_2016} further found that teachers' awareness and perception of their online ratings influenced their sense of professional competence and perceived effectiveness in the classroom. Negative reviews, in particular, were associated with decreased teacher self-efficacy, which could alter teaching practices and pedagogical choices over time. Notably, the literature on this topic is sparse, with the discussion here primarily based on the findings from the study by \textcite{boswell_ratemyprofessors_2016}. Further research is warrented to expand our understanding on the impact of RateMyProfessor.com on professors and educators.

\section{Discussion}
This literature review illustrates how RateMyProfessor is not only widely used by students but also influences their course selections and perceptions of courses and instructors. We discuss a few of the primary effects observed from the website, with a detailed focus on 
platform engagement, self-efficacy, and decision-making. Existing research provides adequate evidence to support these impacts both in and out of the classroom. Together, these impacts manifest in the way students approach their academic journey, potentially guiding their educational experience from information gathered on the platform. 

\subsection*{Impacts on Students}
The impacts of RateMyProfessor.com on student behavior and decision-making is more extensively studied. \textcite{scherr_single_2013} reveals the ability of a single negative comment to shift student perception despite an overall positive rating, demonstrating the potency and weight of the platform in academia. It is apparent from recent studies that the platform is broadly utilized by students as a tool for avoiding the "bad" professors and selecting the "good" professors during their course selection. This behavior closely aligns with the intent of the platform to help students make "critical education choices". On this basis, the platform serves its intended purpose, however, given it could have implications for professors as well it could be under serving a portion of its user base.

\subsection*{Impacts on Professors}
The effects of RateMyProfessor.com on educators is notably underexplored.  Insights from \textcite{boswell_ratemyprofessors_2016} reveals the potential influence of RateMyProfessor evaluations on professors' self-efficacy. These findings suggest the platform may carry broader implications for educators, potentially shaping their teaching methods or styles. Furthermore, how these impacts might vary across different academic disciplines, institutional settings, geographic regions, or cultural contexts remains an open question that warrants future research.

\subsection*{Suggestions for Platform Enhancement}
Considering the potential implications of RateMyProfessor for educators, the platform could enhance its effectiveness by implementing modifications and introducing features that specifically address the needs and concerns of professors. Potential enhancements could encompass visualizing professor feedback over time to reflect growth and change, implementing student verification for credibility, enhancing transparency of the rating process, and providing more opportunities for professor engagement on the platform. 

In addition to the benefits to educators, these proposed enhancements to RateMyProfessor also promise significant advantages for students, leading to a more balanced and informative platform for all users. Given that the platform is commonly used as a tool for course selection, students stand to benefit from additional insights that allow for better judgement of a professor beyond their quality ratings. Potential features that could aide students in evaluating professor profiles could include tools for comparison both within a single department and across different departments, alongside visualizations that contrast individual quality ratings with university-wide averages.

\section{Summary}
This literature review has explored the multifaceted role of RateMyProfessor.com in shaping the academic journey of both students and educators in higher education. Key findings reveal that the platform has a significant ability to influence course selection choices for students, alongside general perceptions about their instructors. Furthermore, studies reveal that exposure to online professor evaluations can impact students' self-efficacy, expectations for course environment, and expectations of success. The effect of RateMyProfessor.com on educators, although less extensively studied, emerges as another area of interest. Despite professors viewing RateMyProfessor evaluations as less credible than university evaluations, it was revealed that exposure to RateMyProfessor reviews influenced their self-efficacy and perception of professional competence.

Proposed enhancements to the RateMyProfessor platform, aimed at addressing the specific needs for both students and teachers, have been discussed. These improvements include introducing features that would allow a better perspective of professor evaluations over time, implementing student verification processes for credibility, and offering comparative tools for better evaluation of educators. For students, these features could provide deeper insights into professor profiles, aiding in more informed decision-making. For educators, these enhancements could offer a more accurate reflection of their teaching effectiveness and growth over time. Overall, these modifications aim to enhance the platform's effectiveness, reliability, and utility to both students and educators.

In summary, it is evident that RateMyProfessor serves as a pivotal tool in higher education. The platforms impact on students' course-selection, teacher perceptions, and anticipated success highlight the necessity for ongoing research and thoughtful development. While it is clear that RateMyProfessor appears to be here to stay for the foreseeable future, enhancing the platforms capabilities to better align with the needs of both students and teachers could allow for a more balanced, equitable, and insightful educational environment.

\printbibliography
\end{document}