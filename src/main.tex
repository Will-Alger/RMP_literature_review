\documentclass[doc,12pt, twocolumn]{apa7}

\usepackage[utf8]{inputenc}
\usepackage[english]{babel}
\usepackage[style=authoryear,backend=bibtex]{biblatex}
\usepackage{csquotes}
\usepackage{ragged2e}
\usepackage{microtype}

\addbibresource{references.bib}

\title{Exploring the Impacts of Ratemyprofessor.com on Higher Education: A Literature Review}
\shorttitle{Ratemyprofessor: A Literature Review}
\author{William Alger}
\affiliation{Northern Kentucky University}
\note{Special Note (Optional)}

\begin{document}
\maketitle

\section{Abstract}
\begin{abstract}
-This literature review examines the impact of Ratemyprofessor.com, a popular online academic review platform, on higher education. The paper synthesizes existing research to understand the site's influence on students' academic decisions and perceptions of educators, as well as its potential biases and limitations. The study highlights the site's enduring popularity amongst students, with substantial use for decision-making regarding course and instructor choices. However, despite widespread usage, questions regarding the platform's validity and reliability persist due to unverified reviews. This review aims to clarify Ratemyprofessor.com's multifaceted role within academia and identify areas for future investigation.
\end{abstract}

\section{Introduction}
The rise of digital platforms has drastically altered many aspects of our life, including how students make academic decisions and shape their college experience. One of the most popular and widely utilized websites, Ratemyprofessor.com, has become an influential tool for students to voice their experiences and opinions on their educators. This literature review aims to explore the comprehensive research associated with Ratemyprofessor.com. It seeks to present an overview of the existing scholarship while pinpointing areas that warrant further exploration. The goal is to offer a thorough understanding of the platform's reach, its insights, and the yet-to-be investigated aspects of its influence.

\subsection{Overview of Ratemyprofessor.com}
Rate my professor is a website that has been around since 1999. For the longest time, it has been the leading public platform for students to anonymously leave reviews about their college professors. Many competitors have come and gone while Ratemyprofessor.com has stood the test of time. Despite its popularity, professors, researchers, and students alike have often debated on its true value. This is in part due to the fact that the site does not require a reviewer to verify that they took the class, nor that they are even a college student. Despite its limitations, the vast amount of publicly available data - with a little over two million professor profiles - has drawn interest among researchers seeking to understand student perceptions in academia. This contrasts with formal university evaluations, which are frequently difficult to obtain and inaccessible for research purposes.

\section{Review of literature}
In the following section, our review of literature will explore the array of research that has been conducted around Ratemyprofessor. This synthesis aims to expose common themes and findings from numerous studies on the platform over the years. Several dimensions of this topic are worth exploring: the impact this site has had on educators and their teaching practices, the influence it exerts on students' course and instructor choices, potential biases inherent in the reviews, and other factors at play. By presenting a comprehensive overview of these aspects and revealing unexplored areas of research, we aim to both bring clarity to the multifaceted role Ratemyprofessor plays within academia and provide direction for future investigations.

\subsection{Effect of Ratemyprofessor on Students}
% Usage of Ratemyprofessor
Numerous studies have consistently highlighted the widespread usage of RateMyProfessor among students. A 2009 study by \textcite{davison_how_2009} surveyed 216 students, revealing that 92\% were aware of RateMyProfessor, 80\% had visited the site more than once, and a notable 95\% deemed it a credible source of information. Further supporting these findings, \textcite{bleske-rechek_ratemyprofessors_2010} showed that among their survey of 208 respondents, 84\% of students had visited the site and 23\% had posted a review. More recently, a study conducted by \textcite{chiang_students_2017} with 166 students from a marketing class demonstrated the continued usage of RateMyProfessor: 84.4\% of these students had visited the site within the past two years, and nearly a quarter (24.6\%) had actively participated by posting ratings. It is clear that Ratemyprofessor has not only drawn the interest of students, but continued to maintain its popularity throughout the years since its launch.

% Usage as a tool for decision making
With such widespread usage of RateMyProfessor demonstrated, researchers have been prompted to explore the extent to which this platform shapes student perceptions of instructors and guides their academic decisions. In a survey of 73 participants, \textcite{boswell_effects_2020} concluded that exposure to either university SETs (Student Evaluations of Teaching) or Ratemyprofessor evaluations could influence students choice to enroll in a course at all. This is supported by \textcite{johnson_i_2014} who found that course enrollment was positively correlated with professor ratings.

% Effect on student self efficacy
The influence of professor ratings on students extends beyond course selection to their anticipated success in class. \textcite{boswell_effects_2020} discovered that evaluations, either positive or negative, significantly influenced students' self-efficacy, thereby affecting their confidence in the course content and expected learning outcomes. This finding is echoed by \textcite{kowai-bell_rate_2011}, which demonstrated a positive influence of Ratemyprofessor evaluations on students' perceived control, grade expectancy, and attitude toward the class.

Moreover, as \textcite{scherr_single_2013} highlighted, even a single negative comment on Ratemyprofessor can significantly impact students' perception of the course climate, despite accompanying positive ratings. Therefore, these studies collectively suggest that the evaluations on Ratemyprofessor play an integral role not only in course selection but also in shaping students' attitudes and expectations towards their chosen classes.


% \textcite{kowai-bell_rate_2011}
% \textcite{scherr_single_2013}








\printbibliography
\end{document}