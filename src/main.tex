\documentclass[man,12pt, twocolumn]{apa6}

\usepackage[utf8]{inputenc}
\usepackage[english]{babel}
\usepackage[style=authoryear,backend=bibtex]{biblatex}
\usepackage{csquotes}
\addbibresource{references.bib}

\title{Unpacking Student Perceptions: A Review of Research on Ratemyprofessor.com}
\shorttitle{Ratemyprofessor: A Research Review}
\author{William Alger}
\affiliation{Northern Kentucky University}
\note{Special Note (Optional)}


\abstract{
    \begin{abstract}
This is a sample abstract for your literature review. An abstract typically includes information about the purpose of the research, the main findings or results, and the conclusions drawn from these results. It succinctly summarizes the key points of your paper.
\end{abstract}
}

\begin{document}
\maketitle

\section{Introduction}
The rise of digital platforms has drastically altered many aspects of our life, 
including how students make academic decisions and shape their college experience.
One of the most popular and widely utilized websites, Ratemyprofessor.com, has become
an influential tool for students to voice their experiences and opinions on their educators. This literature review aims to explore the comprehensive research associated with Ratemyprofessor.com. 
It seeks to present an overview of the existing scholarship while pinpointing areas that warrant further exploration.
The goal is to offer a thorough understanding of the platform's reach, its insights, and the yet-to-be investigated 
aspects of its influence.
\subsection{Overview of Ratemyprofessor.com}
Rate my professor is a website that has been around since 1999. For the longest time,
it has been the leading public platform for students to anonymously leave reviews about 
their college professors. Many competitors have come and gone while Ratemyprofessor.com
has stood the test of time. Despite it's popularity, professors, researchers, and students
alike have often debated on its true value. This is in part due to the fact that the site 
does not require a reviewer to verify that they took the class, nor that they are even a 
college student. Despite its limitations, the vast amount of publicly available data - 
with a little over two million professor profiles - has drawn interest among researchers seeking to
understand student perceptions in academia. This contrasts with formal university evaluations, which are
frequently diffucult to obtain and inaccessible for research purposes.


\section{Review of literature}
\subsection{Effect of Ratemyprofessor on Students}
This is a test citation \textcite*{lewandowski_just_2012} 






\printbibliography
\end{document}